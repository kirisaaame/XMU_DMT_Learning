\documentclass[11pt]{article}

\usepackage[a4paper]{geometry}
\geometry{left=2.0cm,right=2.0cm,top=2.5cm,bottom=2.5cm}

\usepackage{comment}
\usepackage{booktabs}
\usepackage{graphicx}
\usepackage{diagbox}
\usepackage{amsmath,amsfonts,graphicx,amssymb,bm,amsthm, mathtools}
\usepackage{algorithm,algorithmicx}
\usepackage[noend]{algpseudocode}
\usepackage{fancyhdr}
\usepackage{tikz}
\usepackage{graphicx}
\usetikzlibrary{arrows,automata}
\usepackage{hyperref}
\usepackage{soul}
\usepackage{physics}
\usepackage{ctex}
\setlength{\headheight}{14pt}
\setlength{\parindent}{0 in}

\newtheorem{theorem}{Theorem}
\newtheorem{lemma}[theorem]{Lemma}
\newtheorem{proposition}[theorem]{Proposition}
\newtheorem{claim}[theorem]{Claim}
\newtheorem{corollary}[theorem]{Corollary}
\newtheorem{definition}[theorem]{Definition}
\newtheorem*{definition*}{Definition}

\newenvironment{question}[2][Question]{\begin{trivlist}
\item[\hskip \labelsep {\bfseries #1}\hskip \labelsep {\bfseries #2.}]}{\hfill$\blacktriangleleft$\end{trivlist}}
\newenvironment{answer}[1][Answer]{\begin{trivlist}
\item[\hskip \labelsep {\bfseries #1.}\hskip \labelsep]}{\hfill$\lhd$\end{trivlist}}

\newcommand\E{\mathbb{E}}
\newcommand{\cov}{\operatorname{cov}}
\newcommand{\RR}{\mathbb{R}}


\title{Homework Set \#8}
\usetikzlibrary{positioning}

\begin{document}

    \pagestyle{fancy}
    \lhead{Peking University}
    \chead{}
    \rhead{GAMES 001, 2024 Spring}

    \begin{center}
        {\LARGE \bf Homework \#8}\\
        {Due: 2024-7-2 00:00 \quad$|$\quad 6 Questions, 100 Pts}\\
        {Name: XXX}
    \end{center}
    
    \begin{question}{1 (25') (曲率、挠率与 Frenet 标架)}~\\
    求下列曲线的曲率和挠率:
    \begin{enumerate}
        \item [a (5')] $\vb*{r}(t) = \qty(at, \sqrt{2}a\log{t}, a/t), \quad a > 0;$
        \item [b (5')] $\vb*{r}(t) = \qty(a(t - \sin{t}), a(1-\cos{t}), bt), \quad a > 0;$
        \item [c (5')] $\vb*{r}(t) = (\cos^3{t}, \sin^3{t}, \cos{2t}).$
    \end{enumerate}

    我们为曲线 $\vb*{r} = \vb*{r}(s)$,其中 $s$ 为弧长参数,得出 Frenet 标架为 $\qty{\vb*{r}(s); \vb*{\alpha}(s), \vb*{\beta}(s), \vb*{\gamma}(s)}.$
    \begin{enumerate}
        \item [d (5')] 假定曲线的挠率 $\tau \ne 0$ 为一个常数,求曲线
        \[ \tilde{\vb*{r}}(s) = \frac{1}{\tau}\vb*{\beta}(s) - \int{\vb*{\gamma}(s) \dd{s}} \]
        的曲率和挠率。
        \item [e (5')] 假定曲线的曲率 $\kappa \ne 0$ 为一个常数,挠率 $\tau > 0,$ 求曲线
        \[ \tilde{\vb*{r}}(s) = \frac{1}{\kappa}\vb*{\beta}(s) + \int{\vb*{\alpha}(s) \dd{s}} \]
        的曲率和挠率,以及它的 Frenet 标架 $\qty{\tilde{\vb*{r}}(s); \tilde{\vb*{\alpha}}(s), \tilde{\vb*{\beta}}(s), \tilde{\vb*{\gamma}}(s)}.$
    \end{enumerate}
    
    \end{question}

    

    \begin{question}{2 (15') (参数曲线)}~\\
    假定 $\vb*{r} = \vb*{r}(s)$ 是以 $s$ 为弧长参数的正则参数曲线,它的挠率不为 $0$,曲率不是常数,并且下面的关系式成立:
    \[ \qty(\frac{1}{\kappa(s)})^2 + \qty(\frac{1}{\tau(s)} \dv{s}\qty(\frac{1}{\kappa(s)}))^2 = R_0^2 = \mathrm{const}, \]
    请证明该曲线落在一个球面上。
    
    \end{question}

    \begin{question}{3 (30') (第一基本形与变换)}~\\
    在球面 $\Sigma: x^2 + y^2 + z^2 = 1$ 上,取 $N = (0, 0, 1), S = (0, 0, -1).$ 对于赤道平面上的任意一点 $p = (u, v, 0),$ 可以作唯一的一条直线经过 $N, p$ 两点,它与球面有唯一的交点,记为 $p'.$
    \begin{enumerate}
        \item [a (5')] 证明:点 $p'$ 的坐标是
        \[ x = \frac{2u}{u^2+v^2+1}, \quad y = \frac{2v}{u^2+v^2+1}, \quad z = \frac{u^2+v^2-1}{u^2+v^2+1}, \]
        并且它给出了球面上去掉北极 $N$ 的剩余部分的正则参数表示。
        \item [b (5')] 求球面上去掉南极 $S$ 的剩余部分的类似地正则参数表示。
        \item [c (5')] 求上面两种正则参数表示在公共部分的参数变换。
        \item [d (5')] 证明球面是可定向曲面。
    \end{enumerate}

    接下来我们要寻找保长对应与保角对应。
    \begin{enumerate}
        \item [e (5')] 证明在悬链面
        \[ \vb*{r} = (a \cosh{t}\cos{\theta}, a \cosh{t}\sin{\theta}, at), \quad -\infty < t < \infty, \quad 0 \leq \theta \leq 2\pi \]
        和正螺旋面
        \[ \vb*{r} = (v \cos{u}, v \sin{u}, au), \quad 0 \leq u \leq 2\pi, \quad -\infty < v < \infty \]
        之间存在保长对应,其中常数 $a > 0.$
        \item [f (5')] 请建立旋转面
        \[ \vb*{r} = \qty(f(u)\cos{v}, f(u)\sin{v}, g(u)) \]
        和平面之间的保角对应。
    \end{enumerate}
    
    \end{question}

    \begin{question}{4 (10') (第三基本型)}~\\
    定义曲面的第三基本型为 $\dd\vb*{n}\cdot\dd\vb*{n}$. 证明:
    \[ \dd\vb*{n}\cdot\dd\vb*{n} + 2H\dd\vb*{r}\cdot\dd\vb*{n}+K\dd\vb*{r}\cdot\dd\vb*{r}=0\text{.}\]
    \end{question}

    \begin{question}{5 (10') (可展曲面)}~
    \begin{enumerate}
        \item [a (5')] 证明:没有平点的曲面$\vb*{r}:D\rightarrow \mathbb{R}^3$是可展的当且仅当$K\equiv0$.
        \item [b (5')] 试构造一个$K\equiv0$的曲面,但它不是可展曲面。
    \end{enumerate}
    \end{question}

    \begin{question}{6 (10') (极小曲面)}~\\
    定义曲面$\vb*{r}:D\rightarrow \mathbb{R}^3$为\textbf{极小曲面}当且仅当$H\equiv0$.
    \begin{enumerate}
        \item [a (5')] 
        考虑由悬链线旋转得到的旋转曲面,即悬链面:
        \[ \vb*{r} = (c\cosh{\frac{u}{c}}\cos{v},c\cosh{\frac{u}{c}}\sin{v},u) \text{.}\]
        证明:悬链面是唯一的既是旋转曲面又是极小曲面的曲面。
        \item [b (5')] [伯恩斯坦定理]证明:如果$\vb*{r} = (u, v, z(u, v))$在$(u, v)\in\mathbb{R}^2$上都有定义且是极小曲面,则$z$一定是线性函数。换句话说如果一个极小曲面是一个平面上的函数图像,则这个极小曲面是个平面。
        % \item [c (?')] [普拉托问题] 如果写了就到第三页了,所以不出了
    \end{enumerate}
    \end{question}

\end{document}
\documentclass[11pt]{article}

\usepackage[a4paper]{geometry}
\geometry{left=2.0cm,right=2.0cm,top=2.5cm,bottom=2.5cm}

\usepackage{comment}
\usepackage{booktabs}
\usepackage{graphicx}
\usepackage{diagbox}
\usepackage{amsmath,amsfonts,graphicx,amssymb,bm,amsthm}
%\usepackage{algorithm,algorithmicx}
\usepackage[ruled]{algorithm2e}
\usepackage[noend]{algpseudocode}
\usepackage{fancyhdr}
\usepackage{tikz}
\usepackage{graphicx}
\usetikzlibrary{arrows,automata}
\usepackage{hyperref}
\usepackage{physics}
\usepackage{ctex}

\makeatletter
\newcommand{\rmnum}[1]{\romannumeral #1}
\makeatother

\setlength{\headheight}{14pt}
\setlength{\parindent}{0 in}

\newtheorem{theorem}{Theorem}
\newtheorem{lemma}[theorem]{Lemma}
\newtheorem{proposition}[theorem]{Proposition}
\newtheorem{claim}[theorem]{Claim}
\newtheorem{corollary}[theorem]{Corollary}
\newtheorem{definition}[theorem]{Definition}

\newenvironment{question}[2][Question]{\begin{trivlist}
\item[\hskip \labelsep {\bfseries #1}\hskip \labelsep {\bfseries #2.}]}{\hfill$\blacktriangleleft$\end{trivlist}}
\newenvironment{answer}[1][Answer]{\begin{trivlist}
\item[\hskip \labelsep {\bfseries #1.}\hskip \labelsep]}{\hfill$\lhd$\end{trivlist}}

\newcommand\E{\mathbb{E}}

\def\onedot{$\mathsurround0pt\ldotp$}
\def\cddot{% two dots stacked vertically
  \mathbin{\vcenter{\baselineskip.67ex
    \hbox{\onedot}\hbox{\onedot}}%
  }}%
\def\cdddot#1{% three dots 
  \mathbin{\vcenter{\baselineskip.67ex
    \hbox{\onedot}\hbox{\onedot}\hbox{\onedot}%
  }}%
}


\title{Homework Set \#4}
\usetikzlibrary{positioning}

\begin{document}

    \pagestyle{fancy}
    \lhead{}
    \chead{}
    \rhead{GAMES 001, 2024 Spring}

    \begin{center}
        {\LARGE \bf Homework \#5}\\
        {Due: 2024-5-28 00:00 \quad$|$\quad 1 Questions, 100 Pts}\\
        {Name: XXX}
    \end{center}

    \begin{question}{1 (100') (梳状函数)}~
    如课件26-28页所示,梳状函数被定义为:
    \begin{equation*}
        C_T(t) = \sum_{n=-\infty}^{\infty}\delta(t-nT)
    \end{equation*}
    \begin{enumerate}
        \item 试证明周期为$T$的梳状函数的傅里叶变换是周期为$\frac{2\pi}{T}$的梳状函数,即:
    \begin{equation*}
        \mathcal{F}(C_T(t)) = \frac{1}{T}C_{\frac{2\pi}{T}}(\omega).
    \end{equation*}
    \item 试证明任意函数$F(\omega)$卷积梳状函数$C_{\frac{2\pi}{T}}(\omega)$等价于将$F(\omega)$每隔$\frac{2\pi}{T}$平移复制一份后再叠加。

    \end{enumerate} 
    
    \end{question}

\end{document}
\documentclass[11pt]{article}

\usepackage[a4paper]{geometry}
\geometry{left=2.0cm,right=2.0cm,top=2.5cm,bottom=2.5cm}

\usepackage{comment}
\usepackage{booktabs}
\usepackage{graphicx}
\usepackage{diagbox}
\usepackage{amsmath,amsfonts,graphicx,amssymb,bm,amsthm, mathtools}
\usepackage{algorithm,algorithmicx}
\usepackage[noend]{algpseudocode}
\usepackage{fancyhdr}
\usepackage{tikz}
\usepackage{graphicx}
\usetikzlibrary{arrows,automata}
\usepackage{hyperref}
\usepackage{soul}
\usepackage{physics}
\usepackage{ctex}
\setlength{\headheight}{14pt}
\setlength{\parindent}{0 in}

\newtheorem{theorem}{Theorem}
\newtheorem{lemma}[theorem]{Lemma}
\newtheorem{proposition}[theorem]{Proposition}
\newtheorem{claim}[theorem]{Claim}
\newtheorem{corollary}[theorem]{Corollary}
\newtheorem{definition}[theorem]{Definition}
\newtheorem*{definition*}{Definition}

\newenvironment{question}[2][Question]{\begin{trivlist}
\item[\hskip \labelsep {\bfseries #1}\hskip \labelsep {\bfseries #2.}]}{\hfill$\blacktriangleleft$\end{trivlist}}
\newenvironment{answer}[1][Answer]{\begin{trivlist}
\item[\hskip \labelsep {\bfseries #1.}\hskip \labelsep]}{\hfill$\lhd$\end{trivlist}}

\newcommand\E{\mathbb{E}}
\newcommand{\cov}{\operatorname{cov}}


\title{Homework Set \#3}
\usetikzlibrary{positioning}

\begin{document}

    \pagestyle{fancy}
    \lhead{Peking University}
    \chead{}
    \rhead{GAMES 001, 2024 Spring}

    \begin{center}
        {\LARGE \bf Homework \#3}\\
        {Due: 2024-6-18 23:59 \quad$|$\quad 5 Questions, 100 Pts}\\
        {Name: XXX, ID: XXX}
    \end{center}

    \begin{question}{1 (15') (Coupon Collector)}~\\
    假设有 $n$ 种抽奖券。每次抽取的过程中抽中任一种奖券的概率均相同。

    \begin{itemize}
        \item[a (5')] 假设抽取了 $X$ 次后第一次抽到第一种抽奖券,求关于 $X$ 的期望 $\E[X]$。
        \item[b (5')] 假设抽取了 $Y$ 次后集齐了全部的奖券,求关于 $Y$ 的期望 $\E[Y]$。
        \item[c (5')] 为了以高于 $1 - \epsilon$ 的概率集齐全部的奖券,求证最少进行抽取的次数
        \[m = \order{n \log{\frac{n}{\epsilon}}}.\]
    \end{itemize}
    \end{question}

    \begin{question}{2 (30') (独立性)}~\\
    在概率论的实践中,我们强调变量间的独立性。对于随机变量 $X, Y$,我们称他们统计上独立,当且仅当他们的联合概率等于它们概率的乘积,即
    \[ \Pr[X \cap Y] = \Pr[X] \Pr[B]. \]
    \begin{itemize}
        \item[a (5')] 请证明对于统计上独立的随机变量 $X, Y$,对于期望算子 $\E{}$ 满足
        \[ \E[X^nY^m] = \E[X^n]\E[Y^m]. \]
    \end{itemize}
    两个变量 $X, Y$ 间的协方差 $\cov{}$ 被定义为
    \[ \cov[X, Y] = \E[XY] - \E[X]\E[Y]. \]
    \begin{itemize}
        \item[b (5')] 从 $[-1, 1]$ 上随机均匀采样,将采样结果作为随机变量 $X$,定义依赖于 $X$ 的随机变量 $Y = X^2$。请计算 $X$ 与 $Y$ 的协方差。
        \item[c (5')] 设随机变量 $X, Y$ 的联合分布密度为
        \begin{equation*}
            f(x, y) = \begin{dcases}
                \frac{1}{4}(1 + xy), & \abs{x} < 1, \abs{y} < 1,\\
                0, & \mathrm{else}.
            \end{dcases}
        \end{equation*}
        请计算判断 $X$ 与 $Y$ 是否独立,$X^2$ 与 $Y^2$ 是否独立。
    \end{itemize}
    
    我们指出,对于随机变量 $X, Y$,若对于任意给定的常数 $a, b$,两者的线性组合 $aX + bY$ 均可表示为单变量正态分布,那么如果 $X, Y$ 不相关,那么它们是独立的。但是实践中,有些人会认为两个线性不相关、正态分布的随机变量一定是统计独立的;有些人会认为正态分布关于随机变量的线性组合是正态分布的。以下两题为两个反例。
    \begin{itemize}
        \item[d (5')] 取 $X$ 为一个期望为 $0$、方差为 $1$的满足正态分布的变量。$W$ 独立于 $X$,以相同的概率取 $1$ 或者 $-1$。计算随机变量 $Y = WX$ 的协方差,并给出例子来说明 $X, Y$ 不独立。
        \item[e (5')] 取 $X$ 为一个期望为 $0$、方差为 $1$的满足正态分布的变量。取
        \begin{equation*}
            Y = \begin{dcases*}
                X, & if $\abs{X} \leq c$, \\
                -X, & if $\abs{X} > c$,
            \end{dcases*}
        \end{equation*}
        其中 $c$ 为某一常数。观察可知若 $c \to 0$,那么 $\cov[X, Y] \to -1$;若 $c \to \infty$,那么 $\cov[X, Y] \to 1$。由于相关性关于 $c$ 连续,那么必然存在一个值 $c$ 使得 $\cov[X, Y] = 0$。请证明 $Y$ 为一个正态分布。
    \end{itemize}
    
    对于一列相同概率空间上的随机变量 $\qty{X_i}_{i=1}^n$,我们称每 $k$ 个元素独立为对于每一个大小不大于 $k$ 的子集 $I \subseteq \qty{1, \ldots, n},\ \abs{I} \leq k$,对于任一取值列 $\qty{a_i}$,满足
    \[ \Pr[\bigwedge_{i \in I} X_i = a_i] = \prod_{i \in I} \Pr[X_i = a_i]. \]
    若 $k=n$,则这些 $\qty{X_i}$ 相互独立。
    \begin{itemize}
        \item[f (5')] 设三维随机向量 $(X, Y, Z)$ 的联合密度函数为
        \begin{equation*}
            f(x, y, z) = \begin{dcases}
                \frac{1}{8\pi^3}\qty(1 - \sin{x}\sin{y}\sin{z}), & 0 < x, y, z < 2\pi, \\
                0, & \mathrm{else}.
            \end{dcases}
        \end{equation*}
        请求出 $X, Y, Z$ 各自的边际分布,并判断是否两两独立?是否相互独立?
    \end{itemize}
    \end{question}

    \begin{question}{3 (10') (大数定律)}~\\
    尽管很多情形下大数定理可以得到满足,但是我们也要注意不满足的情形。
    \begin{itemize}
        \item [a (5')] 令 $\qty{X_n, n \geq 2}$ 为一列独立的随机变量序列,满足
        \[ \Pr[X_n = \pm n] = \frac{1}{2n\log n}, \Pr[X_n = 0] = 1 - \frac{1}{n \log n}, n = 2, 3, \ldots \]
        请证明 $\qty{X_n, n \geq 2}$ 满足弱大数律,不满足强大数律。
        \item [b (5')] 令$\qty{X_n, n \geq 2}$ 为一列独立的随机变量序列,密度函数为
        \[ f_n(x) = \frac{1}{\sqrt{2} \sigma_n} \exp{-\frac{\sqrt{2}\abs{x}}{\sigma_n}}, \quad x \in \mathbb{R}, \]
        其中 $\sigma_n^2 = 2n^2 / (\log n)^2, n \geq 2$。
        请证明 $\qty{X_n, n \geq 2}$ 满足强大数律,不满足弱大数律。
    \end{itemize}
        
    \end{question}

    \begin{question}{4 (15') (鞅)}~\\
        随机过程中一个重要的基础概念是鞅(Martingale)。此处仅作简单介绍,图形学中的应用将在微分方程一节的习题中展示。
        
        取 $(Z_i)_{i=1}^n$ 与 $(X_i)_{i=1}^n$ 为共同的概率空间上的一列随机变量,若是对于所有的 $i$,$\E\qty[X_i \mid Z_1 \ldots Z_{i=1} ] = X_{i-1}$,那么 $(X_i)$ 被称为关于 $(Z_i)$ 的鞅。进一步地,$Y_i = X_i - X_{i-1}$ 被称为鞅差序列,满足对于所有的 $i$ 而言,$\E\qty[Y_i \mid Z_1 \ldots Z_{i=1} ] = 0$。

        鞅无处不在。事实上,对于任意的随机变量我们都可以得到一个鞅。

        \begin{itemize}
            \item [a (5')] 令 $A$ 与 $(Z_i)$ 为共同概率空间上的随机变量。请证明
            \[ X_i = \E\qty[A \mid Z_1 \ldots Z_i] \]
            是一个鞅。
        \end{itemize}

        以上定义对应的鞅被称为关于 $A$ 的 Doob martingale。 

        选取 $\mathcal{F}_i = \qty{Z_1, \ldots, Z_i}$,我们称一个随机变量 $T \in \qty{0, 1, 2, \ldots} \cup \qty{\infty}$ 为一个停时,则事件 $\qty{T = i}$ 关于 $\mathcal{F}_i$ 可测。即,已知 $\mathcal{F}_i$ 以后,$\qty{T = i}$ 是否成立便可知晓,而不依赖此后的历史。例如第 $1$ 次抛硬币朝上为一个停时,而第一次硬币朝下前的最后一次朝上便不构成一个停时。

        若是停时满足 $\E[T] < \infty$ 且对于所有的 $i$ 与某一指定常数 $c$ 满足 $\E\qty[\abs{X_i - X_{i-1} \mid \mathcal{F}_i}] \leq c$,那么可以得到 $\E[X_T] = \E[X_0]$。
        
        \begin{itemize}
            \item [b (5')] 一个赌徒一开始身无分文,他每局以相同的概率赢得一块钱或者输掉一块钱。如果他输了 $a$ 块钱或者赢了 $b$ 块钱便离开,其中 $a, b$ 均为整数。求问他赢得 $b$ 块钱的概率。
            \item [c (5')] 求问他需要多少时间才会离开。请使用 $Y_i = X_i^2 -i$ 做变量代换。
        \end{itemize}
        
        
    \end{question}    
    \begin{question}{5 (30') (代码填空)}~\\
        请完成代码包中给出的任务。
        
    \end{question}    
\end{document}